\begin{comment}
	A Felhasználói dokumentáció tartalmazza
	- a megoldott probléma rövid megfogalmazását,
	- a felhasznált módszerek rövid leírását,
	- a program használatához szükséges összes információt

	Magába foglalja a telepítési- (vagy üzemeltetési-) és a végfelhasználói leírást. Ezek
	meghatározott célközönséghez szólnak, könnyen és gyorsan kell, hogy eligazítsák a
	felhasználót a program használatában!

\end{comment}

\section{Bevezetés}
%% A feladat rövid ismertetése (mire való a szoftver)
%% Célközönség (kik, mikor, mire használhatják a programot)
\section{Telepítési útmutató}
\subsection{Rendszer követelmények}
%% A rendszer használatához szükséges minimális, illetve optimális HW/SW környezet
	A szoftver Linux Mint 17.1 'Rebecca' Cinnamon 64-bit operációs rendszeren lett tesztelve.
	\begin{center}
  	\begin{tabular}{| p{5cm} | p{10cm} |} 
  	\hline
        Operációs rendszer & debian alapú rendszer (ubuntu, mint)
    \\ \hline
        Apache & 2.4.7 (Ubuntu)
    \\ \hline
        Erlang & 5.10.4 (SMP,ASYNC\_THREADS) (BEAM) emulator 
    \\ \hline
        g++ & 4.8.2 (Ubuntu)
    \\ \hline
        Mozilla Firefox & 37.0.2
    \\ \hline
    
    \end{tabular}
    \end{center}

    Verzió telepítések ajánlott módja terminálból:
	\begin{verbatim}
	sudo apt-get update

	sudo apt-get install g++-4.8
	sudo update-alternatives --install /usr/bin/g++ g++ /usr/bin/g++-4.8 20
	sudo apt-get install erlang
	sudo apt-get install apache2
	sudo apt-get install libapache2-mod-proxy-html
	sudo apt-get install libxml2-dev

	sudo apt-get upgrade
	\end{verbatim}
\subsection{Rendszer konfiguráció}
Első lépésként a forráskódat másoljuk át a gépre.
Terminálban menjünk a mappába.
\begin{verbatim}
user@computername:[project](master)
\end{verbatim}
Apache szerver konfigurációt elég csak a szerver gépen futtatni, a segéd számító gépeken nem szükséges.
\subsubsection{Apache szerver - szakdoli.config}
	A projektben található /project/ServerConfig/szakdoli.conf fájl mintájára létre lehet hozni a saját szerver konfigurációs fájlunkat.
	\newline
	Az elején megadjuk a külső figyelési pontot.
	\begin{verbatim}
	Listen 8086
	<VirtualHost *:8086>
	\end{verbatim}

	Beállíthatjuk a szerver adminját.
	
	\begin{verbatim}
	ServerAdmin webmaster@localhost
	\end{verbatim}

	A mappa helyét \textbf{mindenképpen módosítani kell} a projekt aktuális mappájára, ahova másoltuk.
	DocumentRoot és a Directory után is.
	
	\begin{verbatim}
		DocumentRoot /home/../project/WebPage/
		<Directory /home/../project/WebPage/>
	\end{verbatim}
	
	A szerver elosztott része a 8082 porton van elindítva, és erre hozunk létre egy proxy-t hogy a weboldallal lehetősége legyen kommunikálni.
	
	\begin{verbatim}
		ProxyPass /API http://localhost:8082
		ProxyPassReverse /API http://localhost:8082
		<Proxy *>
		    Order deny,allow
		    Deny from all
		    Allow from all
		</Proxy> 
	\end{verbatim}
	
	Ha valami hiba van a szerveren, a logok segítségével ki lehet deríteni a hiba okát. Ehhez lehetőség van beállítani saját mappát is.

	\begin{verbatim}
		ErrorLog ${APACHE_LOG_DIR}/error.log
		CustomLog ${APACHE_LOG_DIR}/access.log combined
	\end{verbatim}

\subsubsection{Apache szerver - elindítás}
	Az előbbi pontban megvalósított szerver konfigurációs fájt át kell helyezni az apache /sites-available mappábájába, és fel kell venni a konfigurálandó fájlok közé. Ha nincsen még a mód proxy\_http-re állítva, azt is meg kell csinálni a restart előtt. 
	\begin{verbatim}
		sudo cp ./ServerConfig/szakdoli.conf /etc/apache2/sites-available/szakdoli.conf
		sudo a2enmod proxy proxy_http
		sudo a2ensite szakdoli.conf
	\end{verbatim}
	Ezután ha a fájl rendben van el lehet indítani a szervert, mely ezután figyelni fogja az adott portokat.
	\begin{verbatim}
		sudo /etc/init.d/apache2 restart
	\end{verbatim}

\subsubsection{Elosztott számításhoz}
	Az Erlang Node-ok kommunikációjához be kell állítani a .erlang.cookie fájlt mely a /home-ban található. 
	Nem biztos hogy a fájlnak van írás joga. Ha nincs akkor adni kell neki, és meg kell nyitni valamilyen szerkesztőben.
	Szerkesztés után pedig ajánlott vissza adni az eredeti jogait.
	\begin{verbatim}
		sudo chmod +w ~/.erlang.cookie
		sudo vim ~/.erlang.cookie
		sudo chmod 400 ~/.erlang.cookie
	\end{verbatim}
	Az alábbi atomot tartalmaznia kell a fájlnak.
	\begin{verbatim}
	cat ~/.erlang.cookie
	distributed_interpolation_bylexy
	\end{verbatim}
	Ezután az Erlang-node-ok tudnak egymással kommunikálni, akár több gépen is.

\subsubsection{Szerver tesztelése}
	Az első üzembe helyezés előtt érdemes lefordítani a fájlokat, és lefuttatni a teszteket hogy tudjuk hogy az Erlang és a C++ verzió kompatibilis az eredetivel. \newline
	A bin mappában található fájlok segítenek minket a rendszer élesítésében, tehát ebben a mappában van lehetőség a futtatásra.  
	\begin{verbatim}
		$ cd project/bin/
	\end{verbatim}

	Először indítsuk el az alábbi fájlt, mely lefordítja nekünk a kellő részeket. Ennek a fájlnak a futása pár percet igénybe vehet. Ha nem sikerül lefutnia hiba nélkül, akkor nagy valószínűséggel az elosztott számítás nem konfigurálható. 
	\begin{verbatim}
		./setup.sh
	\end{verbatim}
	Ha sikeres volt a futás akkor elindíthatjuk az erlang shell-t, és a run erlang fájl segítségével betölthetjük a lefordított elemeket. A run:load(). futtatását shell indítás után egyszer szabad lefuttatni, mivel a NIF fájlok nem tudnak újra betöltődni, és ez hibát generál.\newline
	Érdemes ismét lefuttatni a teszteket, ha sikeres volt akkor a gép alkalmas arra hogy szerver vagy segéd gép legyen.
	\begin{verbatim}
		erl
		c(run).
		run:load().
		run:test().
	\end{verbatim}
	Ha az erlang fájlokban módosítunk valamit akkor elegendő csak újra indítani a shell-t és lefuttatni a betöltés előtt az Erlang fájlok újra fordítását.
	\begin{verbatim}
		erl
		c(run).
		run:compile().
		run:load().
		run:test().
	\end{verbatim}  
	
%% Első üzembe helyezés leírása – ha van ilyen –, a program indítása (kivéve, ha nem egy
%%önálló alkalmazásról, hanem egy meglévő rendszer új komponenséről van szó). Itt
%%ellenőrizzük, hogy a telepítési útmutató megfelel-e a valóságos telepítési folyamatnak.

\subsection{Használati útmutató}
\subsubsection{Szerver elindítása}
	A szerver elindítását a /project/bin mappában kell végezni. Miután az előző lépésben már a teszteket elvégeztük így elindíthatjuk a szervert és a gépeket.
	Adnunk kell egy nevet a node-unknak, melyet a shell indításakor az -sname kapcsolóval lehet megadni.
	\begin{verbatim}
		$ cd project/bin/
		erl -s toolbar -sname nodeNameForInterpolation
		c(run).
		run:load().
		run:test().
	\end{verbatim}
	Ha a shell-ben minden megfelelően működik (fájlok lefordultak, tesztek lefutottak) akkor incializálhatjuk a portokat, a run:initServer() függvényével. Ez a függvény létrehozza a szerver és a node-figyelő folyamatokat. Kiírja nekünk a képernyőre a figyelő folyamat pid-jét mellyel lehetőségünk van tesztelni, amikor esetleg egy gép felcsatlakozik. 
	\begin{verbatim}
		(nodeNameForInterpolation@computername)4> node().
		nodeNameForInterpolation@computername

		(nodeNameForInterpolation@computername)5> run:initServer().
		WatcherNode : <0.100.0>
		%% ...
	\end{verbatim}
	A számító folyamathoz szükségünk van a másik gépen meghívott node(). eredményére, mivel ez a node incializálójának paramétere. 
	\begin{verbatim}
		(nodeNameForInterpolation@computername2)5> run:initNode(nodeNameForInterpolation@computername).
		<0.66.0>
	\end{verbatim}
	Ekkor ha sikeres volt a feliratkozás akkor a szerver gépen láthatjuk kiírva.
	\begin{verbatim}
		Worker Writed <10214.66.0> 'nodeNameForInterpolation@computername2'
	\end{verbatim}
	A szerveren ezután le lehet futtatni esetleg többször is a tesztet, hogy tudjuk a közös kommunikáció és a tényleges szétosztás is megtörténik.
	\begin{verbatim}
		(test@computername)21> run:test(pid(0,104,0)).
		(test@computername)21> nodeHandler:getNodelist(pid(0,100,0)).
	\end{verbatim}
	Mivel a szerver portjai inicializálódtak, ezért a weboldalról is lehet küldeni a számításokat. 
	A weboldal elérhető az adott porton. A weboldal elérhetővé vált az apache konfiguráció után, és a kommunikáció is működik, ha a szervert inicializáltuk. Az oldal Mozzilla Firefox-ban lett tesztelve. Linkek ilyen típusúak lehetnek: 
	\begin{verbatim}
	http://localhost:8086
	http://<inet addr>:8086
	\end{verbatim}
	Közvetlen szerver kommunikációt az alábbi linken lehet elérni. 
	Ha a kommunikáció kiépül akkor JSON-t kapunk válaszul, ha nem épül ki kapcsolat, akkor más hibaüzenetet kapunk. 
	\begin{verbatim}
	http://localhost:8086/API
	http://<inet addr>:8086/API 
	{"success":"false","msg":"Invalid Response"}
	\end{verbatim}

%% Általános felhasználói tájékoztató (például a szokásostól eltérő képernyő-, billentyű-,
%%illetve egérkezelés leírása, teendők hibaüzenetek esetén stb.).
%% A rendszer funkcióinak ismertetése. A feladat jellegéből fakadóan célszerű lehet ezt
%%folyamatszerűen, képernyőképekkel alátámasztva bemutatni. A funkciókat ajánlatos a
%%felhasználói szintek szerint csoportosítani. Itt vegyük figyelembe, hogy a leírás a
%%fejlesztői dokumentációban meghatározott részfeladathoz illeszkedik-e, az ott
%% meghatározott funkciókat/használati eseteket írja-e le?
%% A rendszer futás közbeni üzenetei (hibaüzenetek, figyelmeztető üzenetek, felszólító üze-
%%netek stb.) és azok magyarázata – az esetleges üzemeltetési teendőkkel együtt. Itt vegyük
%%figyelembe, hogy tartalmaz-e biztonsági, illetve hibaelhárítási előírásokat?
%% Egyéb, a szoftver használatához szükséges információk.